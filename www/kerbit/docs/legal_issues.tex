\documentclass[a4wide, 11pt]{article}
\usepackage{a4, fullpage}

\begin{document}

\title{Legal issues}

\author{Paul Benn \and Daniel Grumberg \and Amin Karamlou \and Saurav Mitra \and Thomas Szyszko }

\date{\today}

% \maketitle

\clearpage

\section{Libraries}

\begin{center}
	\begin{tabular}{ |c|c|c| } 
		\hline
		\textbf{Library} & \textbf{Licence} \\
		\hline 
		Meteor & MIT \\ 
		\hline
		Meteor packages & MIT \\ 
		\hline
		Meteor iOS & Freeware \\ 
		\hline
		Mocha & MIT \\ 
		\hline
		JQuery & MIT \\ 
		\hline
		Bootstrap & MIT \\ 
		\hline
		Materialize & MIT \\ 
		\hline
		Font-Awesome & SIL OFL \\ 
		\hline
		Google Maps & Freeware \\ 
		\hline
	\end{tabular}
\end{center}

\section{Code}

Our code is licenced under the MIT agreement. This means it is free to use, copy, modify, merge, publish, distribute, sublicence and/or sell. Note this includes the right to link it with code using other licences.

\section{Pictures}

All pictures used in the browser version of our application are our property (photoshop projects or original photographs). The iOS application does not contain any media so far.

\section{Other issues}

\subsection{Release version}

If Kerb.It was ever released, contract law would have to apply to all users and we would have to deal with taxes.

\subsection{Testing frameworks and VCS}

Teamcity, Travis, Git, etc. need no further attention as they do not modify the final product.

\end{document}
