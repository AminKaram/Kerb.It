\documentclass[a4wide, 11pt]{article}
\usepackage{a4, fullpage}
\usepackage{graphicx}
\usepackage{hyperref}
\setlength{\parskip}{0.3cm}
\setlength{\parindent}{0cm}

\begin{document}

\title{Technologies}

\author{Paul Benn \and Daniel Grumberg \and Amin Karamlou \and Saurav Mitra \and Thomas Szyszko }

\date{\today}

 % \maketitle

\clearpage

\section{Language choices}

\begin{itemize}
	\item \textbf{HTML} - Using version 5 for standardization and consistency, improved semantics and better features.
	\item \textbf{CSS} - We are using Twitter Bootstrap 3 extended with Materialize's Sass preprocessor capabilities.
	\item \textbf{JavaScript} - Meteor (see below) and underlying Node.js are written in JavaScript, so it makes sense to use it.
	\item \textbf{Swift} - Most viable and widely preferred choice for iOS application programming.
\end{itemize}

\section{Technology choices}

\begin{itemize}
	\item \textbf{Meteor} - Advanced, extensible, reactive JS-based web application framework. Routing is done via Flow Router (reliability), front-end rendering via Blaze (complex templating).
	\item \textbf{MongoDB} - Meteor is configured to use MongoDB by default. JSON documents are also very easy to inspect and test during debug sessions.
	\item \textbf{XCode} - Standard, fully fledged Swift IDE, no real competitors.
\end{itemize}

\section{Build server}

\url{teamcity.kerbit.co.uk} (login required). Builds automatically deploy to live server if all tests pass.\\
Screen-shot:
% \includegraphics{tests}

\end{document}
